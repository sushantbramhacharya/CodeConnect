\chapter{Background Study and Literature Review}

% (20\% of Report Length)

% a. Must be paraphrased without plagiarizing

% b. Must include the base papers\cite{Adhikari2020Dec}, and support the rationale of the project

% c. Must highlight the strengths and shortcomings of the works performed by other authors

\section{Background Study}

We are looking for designs that make out system visually appealing and at the same time have better performance. As this system is mainly for creatives who can share their journey, we need to implement a profile system that shows off their portfolio and resume. Showcasing their skills should be easy so this system mainly focuses on functionalities implementations. We are looking for different tools and techniques for achieving those goals. We are also studying papers, articles, and related books for our project. We are also learning about implementation about messaging system.
The proposed project is to create an app for programmers where they can share their discussions, projects, skills, and perform messaging functions. To develop this app, it is important to understand code collaboration, tools for code sharing, and messaging functions.


\subsection{Literature Review}
Social networks are like groups of people who know each other and interact with each other. The technology helps us study how people are connected to each other and how they talk to each other online. It also helps us understand the things they say and the information they share.\cite{korshunov2014social}\\
In today's competitive job market, organizations strive to identify and attract top talent, and this research investigates the influence of social media on the recruitment process. With the rapid growth of social media usage, it is crucial for organizations to understand effective strategies for attracting the best candidates. The study involved 12 recruiters from various industries, and the findings reveal heavy reliance on platforms like LinkedIn for recruitment purposes. However, the use of Twitter and Facebook for recruitment is comparatively lower. Recruiters need a focused approach when utilizing social media to manage the potential overwhelming volume of work. It is evident that recruiters cannot effectively conduct recruitment activities without leveraging social media tools, but proper training in optimizing social media usage is essential. This study contributes to highlighting the significant impact of LinkedIn on recruitment processes, while also emphasizing that social media is not a one-size-fits-all solution for recruitment challenges.\cite{koch2018impact}\\
In Stack Overflow, A complete profile includes details such as a website URL, location, about me section, profile image, and age. Our analysis revealed that most users do not have a complete profile. However, users with complete profiles tend to have higher reputation scores and provide better quality question and answer posts compared to users with incomplete profiles. This suggests that having a complete profile is beneficial for contributing effectively to the network. Among the profile elements we examined, location and about me have a stronger relationship with user activity and contribution. This research helps us understand which profile elements are important in a Q and A social network and which ones should be prioritized for users to fill out regularly.\cite{adaji2016towards}