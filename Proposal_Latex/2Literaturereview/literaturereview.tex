\chapter{BACKGROUND AND LITERATURE REVIEW}

% (20\% of Report Length)

% a. Must be paraphrased without plagiarizing

% b. Must include the base papers\cite{Adhikari2020Dec}, and support the rationale of the project

% c. Must highlight the strengths and shortcomings of the works performed by other authors

\section{Background Study}

Our designs makes system visually appealing and at the same time have better performance. As this system is mainly for creatives who can share their journey, A profile system that shows off their portfolio and resume need to be implemented. Showcasing their skills should be easy so this system mainly focuses on functionalities implementations. Different tools and techniques for achieving those goals are important. Studying papers, articles, and related books for our project were reserched. Implementation of Messaging System is being studied.
The proposed project is to create an app for creative IT professionals where they can share their discussions, projects, skills, and perform messaging functions. To develop this app, it is important to understand code collaboration, tools for code sharing, and messaging functions.

Here, User Profiles act as digital info of individuals, showcasing their skills, experiences, and interests, while the dynamic Feed delivers a steady stream programming-related discussions to users' homepages. Posts are user-generated content primarily comprising programming code snippets and text-based discourse, fostering a community centered on knowledge exchange. 'Geek' stands as a form of user endorsement for content, similar to 'liking'. Comments in discussions around posts, empowering interaction. The platform enables private Messaging for real-time one-on-one conversations. Authentication ensures secure access by validating user identities through credentials. The Front-End uses user interface and design elements, while the Back-End constructs data management and application behavior. Responsive Design optimizes cross-device usability, while AJAX facilitates server communication without page reloads. jQuery simplifies DOM manipulation and interactions.

\section{Literature Review}
\subsection{Existing System}
Social networks are like groups of people who know each other and interact with each other. The technology helps us study how people are connected to each other and how they talk to each other online. It also helps us understand the things they say and the information they share \cite{korshunov2014social}.\\
\textbf{Social Networking}\\
In today's landscape of electronic media, the concept of social networking has evolved to signify the utilization of the Internet and various Web applications, enabling individuals to communicate in ways that were previously unimaginable. This transformation has been driven by a paradigm shift that extends across our culture, altering how perceiveed and harnessed the potential of the Internet. The current iteration of the Web is markedly distinct from its incarnation a mere decade ago, emphasizing interactivity, user engagement, and real-time connectivity. This evolution has paved the way for the proliferation of social networking and collaborative platforms, capitalizing on this dynamic environment to facilitate connections and interactions among people, regardless of their physical location. In a more abstract sense, the essence of social networking transcends individuality, striving to encompass a diverse array of individuals. The widespread adoption of social networking websites signifies a pivotal progression in human social dynamics, indicating a departure from conventional face-to-face interactions and embracing digitally mediated connections as an integral facet of contemporary societal interactions. This culmination of electronic media's impact and the resultant transformation in the nature of the Internet underscores a fundamental shift in human social interaction, redefining the boundaries of communication, collaboration, and community-building on an unprecedented scale \cite{weaver2008social}.
\\
\textbf{LinkedIn}\\
In today's competitive job market, organizations strive to identify and attract top talent, and this research investigates the influence of social media on the recruitment process. With the rapid growth of social media usage, it is crucial for organizations to understand effective strategies for attracting the best candidates. The study involved 12 recruiters from various industries, and the findings reveal heavy reliance on platforms like LinkedIn for recruitment purposes. However, the use of Twitter and Facebook for recruitment is comparatively lower. Recruiters need a focused approach when utilizing social media to manage the potential overwhelming volume of work \cite{koch2018impact}.\\
\textbf{Stack Overflow}\\
In Stack Overflow, A complete profile includes details such as a website URL, location, about me section, profile image, and age. Our analysis revealed that most users do not have a complete profile. However, users with complete profiles tend to have higher reputation scores and provide better quality question and answer posts compared to users with incomplete profiles. This suggests that having a complete profile is beneficial for contributing effectively to the network. Among the profile elements they examined, location and about me have a stronger relationship with user activity and contribution. This research helps us understand which profile elements are important in a Q and A social network and which ones should be prioritized for users to fill out regularly \cite{adaji2016towards}.\\
\textbf{AJAX}\\
The term "Ajax" stands for Asynchronous JavaScript and XML, presenting a powerful model that empowers to initiate server requests seamlessly from the client-side code (JavaScript). By adopting this approach, server interactions without obstructing the user's interaction with the web page can trigger. This translates to the ability to update specific parts of the page without necessitating a complete reload, thereby significantly enhancing both the performance of your website and the overall user experience. Central to this process is the concept of asynchronous communication, which allows for data exchanges between the client and server to transpire independently. This asynchronous behavior eradicates the need for users to endure lengthy page refreshes, thereby contributing to a more fluid and dynamic interface \cite{paz2013ajax}.\\
\textbf{GitHub}\\
It examine the characteristics of developers involved in Open Source software creation to understand what factors contribute to innovation within the Open Source community. The analysis reveals that having a higher reputation within the community increases the likelihood of attracting collaborators, although developers are also motivated by reciprocity, aligning with the principles of a gift economy. Additionally, it is a significant network effect resulting from standardization, indicating that developers who use popular programming languages in their projects are more likely to collaborate with others. Furthermore, providing additional information, such as a valid URL to the developer's homepage, increases the chances of finding coworkers. These findings can be applied to the broader population of experienced users on platforms like GitHub \cite{celinska2018coding}.\\
\textbf{GitHub Discussions}\\
GitHub has recently introduced a new feature called Discussions, which serves as a platform for developers to ask questions and engage in broader discussions that go beyond specific Issues. Before its widespread availability in December 2020, Discussions underwent testing on selected open source software projects. In order to gain insights into developers' utilization of this innovative feature, their perceptions of it, and its impact on the software development process, they conducted a comprehensive mixed-methods study involving early adopters of GitHub discussions between January and July 2020.Developers perceive GitHub Discussions as a valuable tool; however, they encounter challenges related to topic duplication between Discussions and Issues. This issue poses a concern, as it leads to confusion and redundancy in communication \cite{hata2022github}.
