\chapter{IMPLEMENTATION}


% (20\% of Report Length)

% a. Showcase the output at various intermediate stages of the project pipeline

% b. Use proper data visualizing techniques to present the output

% c. Figures and tables must be accompanied by an explanation
\section{Tools Used}
\textbf{Figma}\\
Figma is a cloud-based design and prototyping tool that empowers teams to collaborate on UI/UX design projects in real-time. It offers a user-friendly interface and powerful features that make it a popular choice among designers. With Figma, designers can create and share interactive prototypes, design components, and design systems. Its cloud-based nature allows for seamless collaboration, enabling multiple team members to work on the same design simultaneously. Figma supports version control, ensuring that design iterations can be easily tracked and managed. \\
\textbf{HTML/CSS}\\
HTML and CSS are two of the most important languages for creating web pages. HTML stands for HyperText Markup Language, and it is used to structure the content of a web page. CSS stands for Cascading Style Sheets, and it is used to control the appearance of a web page.
HTML is a markup language, which means that it is used to mark up text with tags. These tags tell the web browser how to display the text. For example, the h1 tag tells the web browser to display the text as a heading, while the p tag tells the web browser to display the text as a paragraph.
CSS is a style sheet language, which means that it is used to define styles for HTML elements. These styles can control the font, size, color, and other properties of HTML elements.\\
\textbf{MySQL}\\
MySQL is a robust relational database management system that offers a range of features for efficient data storage and retrieval. It supports transactions with ACID properties, ensuring Atomicity, Consistency, Isolation, and Durability. MySQL provides essential capabilities such as automatically updatable views, triggers, foreign keys, and stored procedures, allowing for complex data manipulation and logic implementation. It is compatible with various operating systems, including Windows, Linux, macOS, FreeBSD, and OpenBSD. \\
\textbf{Git/Github}\\
Git is a distributed version control system that is both free and open-source, designed to handle projects of all sizes efficiently and swiftly. It simplifies collaboration by enabling multiple individuals to contribute changes that can be seamlessly merged into a single source. When using Git, the software runs locally on your computer, storing your files and their complete history. Alternatively, you can utilize online hosts like GitHub to store a copy of your files and their revision history.\\
\textbf{PHP}\\
PHP is a server-side scripting language that is used to create dynamic and interactive web pages. It is a free and open-source language that is widely used by web developers. PHP can be used to process form data, generate dynamic content, and connect to databases. It is also used to create content management systems (CMS) and e-commerce platforms.
PHP is a powerful and flexible language that is easy to learn and use. It is a great choice for web developers who want to create dynamic and interactive web pages.\\
\textbf{JavaScript}\\
JavaScript is a client-side scripting language that is used to create interactive web pages. It is a powerful and versatile language that can be used to do a wide variety of things, including adding animation and interactivity to web pages, validating form data, processing user input, making Ajax requests to the server, and creating games and other interactive applications.\\
\textbf{React .js}\\
React.js is a widely-used JavaScript library for creating efficient and reusable user interfaces. It offers a component-based architecture, virtual DOM for improved performance, and supports declarative programming. With a rich ecosystem of libraries and tools, React.js enables developers to build dynamic and responsive applications for both single-page and server-side rendering.\\
% \subsection{Implementation Details of Modules}
% \section{Testing}
% \subsection{Test Cases for Unit Testing}
% \subsection{Test Cases for System Testing}