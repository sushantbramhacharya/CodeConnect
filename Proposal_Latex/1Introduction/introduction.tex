\chapter{INTRODUCTION}
% (20% of Proposal Length)
\pagenumbering{arabic}

% Introduction: (20\% of Report Length)


\section{Introduction}

Code Connect serves as a social media hub tailored to the needs of IT enthusiast, aiming to fill the void of specialized functionalities on current social platforms. This unique platform provides a designated arena where IT professionals can unite, cooperate, and gain access to their own soical network.
\section{Problem Statement}

There are many general social media platforms available, but none of them are specifically designed for IT professionals. This means that IT professionals often have to use general platforms, which can be less effective for networking and collaboration.
Most general social media platforms do not have dedicated spaces for IT professionals to share their resumes. This can make it difficult for IT professionals to get their resumes seen by potential employers.
There are no specific resume management tools available for IT professionals. This means that IT professionals often have to use general resume management tools, which can be less effective for managing IT-related resumes.

There is no specific portfolio management tool available for IT professionals. This means that IT professionals often have to use general portfolio management tools, which can be less effective for managing IT-related portfolios.
IT professionals are often underrepresented in other social media platforms. This can make it difficult for IT professionals to reach a wider audience and connect with other IT professionals.
The challenges listed above can be even more difficult for new aspiring IT professionals. This is because new IT professionals may not have the same level of experience or connections as more experienced IT professionals.
\section{Objectives}
\begin{itemize}
    \item Create a social media having normal functionalities and extra specifically  for creative it professionals.
\end{itemize}
\section{Scope and Limitations}


The app should provide a space for IT professionals to network with each other. This could be done through discussions or Messaging. Networking can help IT professionals to discover jobs, learn new skills, and stay up-to-date on the latest trends.
The app should make it easy for new comers in field of IT.
The app should offer a nice way to showcase their skill and projects.
The app should have code snippets sharing and discussion.
The app should have connection functions for connecting between peers, friends and seniors.
\newline
\newline
The concept of Code Connect holds promise in addressing a specific IT niche, but it must confront challenges. The competitive landscape might involve existing platforms with strong user bases, while persuading IT professionals to transition from established networks demands a compelling value proposition. The task of maintaining content quality and thwarting spam becomes intricate as the platform scales. Sustainable operation hinges on continuous resource allocation for development, maintenance, and support, requiring careful revenue generation that doesn't compromise user experience. Additionally, prioritizing robust privacy and security measures is crucial to alleviate IT professionals' concerns. Overcoming these obstacles necessitates strategic planning, adaptability, and an unwavering commitment to catering to the IT community's distinctive needs.
\section{Potential applications}
\begin{itemize}
    \item Networking and Collaboration: IT professionals can connect with peers, mentors, and industry experts, fostering collaborations on projects, sharing insights, and expanding their professional network.
    \item Skill Development: Code Connect can share codes regarding to their problems and projects.
    \item Knowledge Sharing: Members can share code and resources related to programming languages, frameworks, tools, and best practices.
    \item Problem Solving: People can comment on problems stating solutions.  5. Messaging: They can use code connect messenger to have private conversations.
  \end{itemize}
\section{Originality of Project}

\begin{itemize}
    \item Specialized Platform: Code Connect stands out by being a social media platform specifically designed for IT professionals, acknowledging their distinct requirements and expertise.
    \item Focused Niche: Unlike generic platforms, Code Connect targets the IT community exclusively, demonstrating a focused approach that caters directly to the needs of this specific audience.
    \item Unique Needs: The project pioneers by recognizing and addressing the unique challenges and aspirations that IT professionals encounter in their career paths, setting it apart from more generalized social networks.
    \item Tailored Experience: By tailoring features and content to the IT sector, Code Connect offers an environment that resonates with professionals in this field, enhancing engagement and relevance.
    \item Career Enhancement: Code Connect provides tools like specialized resume and portfolio management, facilitating the effective representation of IT professionals' skills and experiences to potential employers and collaborators.

    \item Empowerment and Growth: Through Code Connect, IT professionals are empowered to connect, learn, collaborate, and grow within a digital sphere that aligns with their expertise, fostering a sense of belonging and advancement.
    
  \end{itemize}

\section{Report Organisation}
The material in this project report is organised into Five chapters. After this introductory chapter introduces the problem topic this project tries to address, chapter 2 contains the literature review of vital and relevant publications, pointing toward a notable project related infromations. Chapter 3 describes the Designs and Analysis of the System for the implementation of this project and models and methods. Chapter 4 provides an overview of Implementation tools, modules used and testing performed in certain unit. Chapter 5 contains conclusion of our project and outcome till now. After Main Report We have APPENDIX A that contains Gantt Chart, Setup and Implementation Guide and last one is for the references