\documentclass{article}
\begin{document}

\section{Methodology}
\subsection{Waterfall Model}
For the development of this project, waterfall model was choosen among all the other as the waterfall methodology is a software development life cycle (SDLC) model that is characterized by a linear, sequential approach to development. In other words, each phase of the waterfall model must be completed before the next phase can begin which makes the workflow of the project well formatted.The waterfall methodology has five phases:
Requirements gathering and analysis,Design,Implementation,Testing,Deployment.\\
\begin{itemize}
    \item Requirements gathering and analysis: In this phase, the project team gathers and analyzes the requirements for the software product. This includes identifying the needs of the users and stakeholders, as well as the functional and non-functional requirements for the software.
    \item Design: In this phase, the project team creates a detailed design for the software product. This includes creating diagrams, specifications, and other documentation that will be used to build the software.
    \item Implementation: In this phase, the project team builds the software product according to the design. This includes coding, testing, and debugging the software.
    \item Testing: In this phase, the project team tests the software product to ensure that it meets the requirements. This includes unit testing, integration testing, and system testing.
    \item Deployment: In this phase, the software product is deployed to production and made available to users.
\end{itemize}

\subsection{Feasibility study}
\subsubsection{Economical Feasibility}
Since the proposed system has a web application and a mobile application supported
on both Android and iOS, we will be using free and open-source cross platform
software development tools such as Flutter, PHP, MySQL and Figma. We will only need some economy for server for hosting.
\subsubsection{Operational Feasibility}
Operational feasibility for the proposed system focuses ease of use. As the system is designed to be interactive, users do not require in-depth knowledge of the mobile app or web app to navigate and utilize its features. The user interface (UI) is specifically designed to be user-friendly, ensuring a smooth and intuitive experience. This approach minimizes the need for extensive training and reduces potential resistance from users.  By considering these factors, the proposed system aims to achieve operational feasibility and provide a positive experience for both users and customers.



\end{document}